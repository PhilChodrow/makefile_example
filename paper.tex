\documentclass{article}
\usepackage{graphicx}
\title{Makefile Example}
\author{Phil}
\date{\today}
\begin{document}
    \maketitle
    
    This is a simple example of multi-language, research-oriented programming using makefiles. 
    The ``analysis'' has the following steps: 
    \begin{enumerate}
        \item Download a data file from the internet using a simple R script.
        \item Do some basic data processing to that file in Python.
        \item Visualize the processed data in R.
        \item Include the results in a paper produced via LaTeX. 
    \end{enumerate}
    To run this code, you will need a machine with \texttt{GNU Make} installed (all *nix machines should work), as well as \texttt{python}, \texttt{R}, and \LaTeX. 
    You also need the the \texttt{pandas} package for \texttt{python} and the \texttt{ggplot2} package for \texttt{R}. 
    
    \section{Benefits of Makefiles}
        A carefully-crafted makefile provides multiple benefits in the context of data-oriented projects. 
        \begin{enumerate}
            \item First, a makefile is a machine-readable description of the dependency structure of a project. 
            It therefore serves as validated documentation of how the parts of your project fit together. 
            \item Second, a makefile allows you to automatically execute your entire pipeline using a single command (usually \texttt{make all}), and then subsequently ``start from scratch'' by deleting all throughputs (usually \texttt{make clean}). 
            \item Finally, a makefile automatically manages dependencies. Say that the data processing step is expensive, and you don't want to do it every time. 
            \texttt{make} will automatically detect when it is necessary to re-do this computation, only doing so if you have deleted the file or changed the script that executes it. 
            This is very convenient! 
            Here's an easy way to try this out. 
            First, delete this line in \texttt{paper.tex} and then run \texttt{make all}, observing what happens. 
            Then, add a comment to \texttt{process.py}, save, and run \texttt{make all}, and observe the different behavior. 
        \end{enumerate}
    
    \begin{figure}
        \includegraphics[width=\textwidth]{figs/bar.png}
    \end{figure}
\end{document}